\chapter{Preface}
Computer-aided design (CAD) is a powerful tool for
spatial investigation and manipulation. Like many 
powerful tools, its usage can be broken down into
simple steps. I hope to take you through those steps
the right way the first time, so that you build good
habits from day one.

\section{Who is this book for?}
This book assumes no fore-knowledge of CAD or engineering
concepts. We'll start from square one. This book is for
individuals who want to manipulate 2-D and 3-D objects. 
User groups that may find this work useful are:

\begin{itemize}
\item Budding mechanical engineering students. Though I've spent
very little time in my engineering career as a designer,
I still CAD regularly.

\item Hobbyists and artists. I've used CAD to help churn through ideas on personal projects where pencil-on-paper
sketches would be sufficient but slow.

\item Parents who, due to the Covid-19 pandemic, can now
add ``teacher'' to their resume. I wish the lessons in
this book had been a part of my early schooling.
\end{itemize}

\section{Why am I writing this book?}

I'm writing this book because:
\begin{itemize}
\item I, like many, was taught to CAD through word of
mouth. I was blessed to have strong mentors, but I 
believe one can build a solid foundation with 
well-designed lessons. Surprisingly, the resource I
envision doesn't seem to exist.

\item Instruction tends toward details, advanced theory,
and excess talk while missing fundamental concepts and,
most important, practice. I aim to create a book that,
like a good coach,
encourages practice, critical thought, and strong
fundamentals.

\item I believe that learning the skills well from 
Day 1 can shorten the learning curve by months. Learning
them haphazardly can build poor habits that later need to
be remodeled.

\item I've consumed my fair share of do-it-yourself
books. I hope to give back to that collection of knowledge.
\end{itemize}

\section{Using this Book}
This book is intended as a series of work-along exercises.
CAD, like most cognitive skills, is best learned with a
bit of listening and a lot of doing. I recommend these
steps to make the most of this book.

\begin{description}
\item[Mouse] Get yourself a mouse with at least two
buttons and a scroll wheel. I use one with a cord.
\item[This Book] Keep this book open as you work. This is
not a ``read-a-chapter-then-do'' type of book.
\item[Monitors] If you've got two monitors, great! I
don't. Find a way to easily view both Onshape and this book, preferably at the same time.
\end{description}

\section{Using Onshape}
This book uses Onshape because it
is free (as in beer), and therefore accessible. You
will need to make an account at (\url{onshape.com}).
Onshape's free accounts include storage space for public 
documents, which means other users can view and copy 
but not edit your documents. Onshape's free subscription
is not the place to develop highly confidential designs.
Perhaps the world needs less highly confidential designs.

\section{CAD Efficiency}
I find the following allow me to smoothly develop and manipulate my models. I recommend building these habits, and only shunning them once you've determined they don't work for you.

\begin{description}
\item[Hotkeys] I'll call out relevant hotkeys when appropriate. Because keystrokes are more practiced and require less precision than mouse clicks, they are faster for repetitive tasks such as defining a sketch. Where a hotkey appears as (\keystroke{shift} \keystroke{s}), this is translated as ``While holding \keystroke{shift}, press \keystroke{s}''.
\end{description}
